% Created 2020-05-21 Чт 18:53
% Intended LaTeX compiler: pdflatex
\documentclass[11pt]{article}
\usepackage[utf8]{inputenc}
\usepackage[T1]{fontenc}
\usepackage{graphicx}
\usepackage{grffile}
\usepackage{longtable}
\usepackage{wrapfig}
\usepackage{rotating}
\usepackage[normalem]{ulem}
\usepackage{amsmath}
\usepackage{textcomp}
\usepackage{amssymb}
\usepackage{capt-of}
\usepackage{natbib}
\usepackage[linktocpage,pdfstartview=FitH,colorlinks,
linkcolor=blue,anchorcolor=blue,
citecolor=blue,filecolor=blue,menucolor=blue,urlcolor=blue]{hyperref}
\usepackage[utf8]{inputenc} % for cyrilics
\usepackage[russian, english]{babel}
\usepackage[T2A]{fontenc}
\author{Дробин М.Е. (МФТИ ГУ)}
\date{\textit{<2020-02-12 Ср>}}
\title{Применение ансамблей алгоритмов в Рекомендательной системе}
\hypersetup{
 pdfauthor={Дробин М.Е. (МФТИ ГУ)},
 pdftitle={Применение ансамблей алгоритмов в Рекомендательной системе},
 pdfkeywords={},
 pdfsubject={},
 pdfcreator={Emacs 26.3 (Org mode 9.1.9)}, 
 pdflang={English}}
\begin{document}

\maketitle
\tableofcontents

\begin{abstract}
\end{abstract}

\section{Введение}
\label{sec:org849ae77}
В этой работе на задаче рекомендации фильма по списку уже просмотренных фильмов предложено несколько алгоритмов матричной факторизанции,
несколько архитектур нейросетей и градиентного бустинга, и несколько ансамблевых моделей. Качество решения сравнивается с уже работающим
алгоритмом в продакшене. Ставится задача улучшить результаты решения этой задачи за счёт использования нейросетевых решений последних лет и
пнсамблирования алгоритмов.
\section{Постановка задачи}
\label{sec:orga4018da}
Мы будем рассуждать в терминах пользователей (users, U) и товаров (items,
I), но все методы подходят для рекомендаций любых объектов. Будем считать, что
для некоторых пар пользователей \(u \in U\) и товаров \(i \in I\) известны оценки \(r_{ui}\) , которые отражают степень заинтересованности пользователя в
товаре. Вычисление таких оценок — отдельная тема. Например, в интернет-магазине заинтересованность может складываться из покупок товара и просмотров его страницы, причём покупки должны учитываться с большим весом. В социальной сети заинтресованность в материале может складываться из времени просмотра, кликов и явного отклика (лайки, репосты); это всё тоже должно суммироваться с различными весами. Не будем сейчас останавливаться на этом вопросе, а перейдём к основной задаче.
Требуется по известным рейтингам \(r_{ui}\) научиться строить для каждого пользо-
вателя u набор из k товаров I(u), наиболее подходящих данному пользователю — то есть таких, для которых рейтинг \(r_{ui}\) окажется максимальным. Самый распространённый подход в данном случае — сформировать признаки, характеризующие пользователя, товар и их взаимодействия, и обучить модель, которая по данным признакам будет предсказывать рейтинг. Это может быть ранжирующая модель, которая сортирует все товары для данного пользователя; может
быть и обычная поточечная модель. Ниже мы рассмотрим некоторые простые методы рекомендаций, оценки которых, как правило, используются в качестве признаков
для итоговой модели
\subsection{Коллаборативная фильтрация}
\label{sec:orgf3563c6}
Методы коллаборативной фильтрации строят рекомендации для пользователя
на основе похожестей между пользователями и товарами
\subsection{Контентные модели}
\label{sec:orgd7ba4d2}
В коллаборативной фильтрации используется информация о предпочтении
пользователей и об их сходствах, но при этом никак не используются свойства самих пользователей или товаров. При этом мы можем обладать дополнительными дан-
ными — например, текстовыми описаниями или категориями товаров, данными из
профиля пользователя. Из этих данных можно сформировать признаковое описание
пары (пользователь, товар) и пытаться предсказывать рейтинг по этим признакам с
помощью каких-либо моделей (линейных, композиций деревьев и т.д.)
\section{Описание данных}
\label{sec:org908fbd6}
Данные взяты с соревнования REKKO CHALLENGE на boosters 26.02.2019 

Перед соревнование данные предобработали:
\begin{itemize}
\item Данные анонимизированы
\item Для обучения представлена лишь некоторая выборка из всех данных по времени, пользователям и контенту
\item Время выражено в абстрактных единицах, для которых сохранено расстояние и отношение порядка
\end{itemize}

В общем, данные разделили на несколько файлов:
\begin{itemize}
\item информация о транзакциях (покупках и просмотрах по подписке) со временем просмотра по каждой транзакции
\item информация об оценках, поставленных пользователями
\item информация о фильмах, добавленных пользователями в «Запомненное»
\item метаинформация о контенте - анонимизированные признаки фильмов и сериалов
\end{itemize}

Доступ к контенту в Okko осуществляется через приложение на телевизоре или смартфоне, либо через веб-интерфейс. Контент можно взять в аренду®, купить(P) или посмотреть по оформленной подписке(S). Организатор соревнования предоставил данные о просмотрах за N дней (N > 60), дополнительно была доступна информация о проставленных рейтингах и добавлениях в закладки. Стоит иметь в виду одну важную деталь, если пользователь посмотрел один фильм несколько раз или несколько серий сериала, то в табличке будет зафиксирована лишь дата последней транзакции и суммарное время потраченное на единицу контента.

Было предоставлено порядка 10 миллионов транзакций, 450 тысяч оценок и 950 тысяч фактов добавлений в закладки по 500 тысячам пользователей.

В выборке содержатся не только активные пользователи, но и пользователи посмотревшие пару фильмов за весь период.

Каталог Okko содержит контент трех типов: фильмы(movie), сериалы(series) и многосерийные фильмы(multipart\textsubscript{movie}), всего 10200 объектов. По каждому объекту был доступен набор анонимизированных атрибутов(attributes) и признаков(feature\textsubscript{1},… ,feature\textsubscript{5}), доступность по подписке, аренде или покупке и длительность. 

\section{Целевая переменная и метрика}
\label{sec:orgd28bef7}
В задаче требовалось предсказать множество контента, который пользователь потребит за следующие 60 дней. Считается, что пользователь потребит контент, если он:

\begin{itemize}
\item Купит его или возьмет в аренду
\item Посмотрит больше половины фильма по подписке
\item Посмотрит больше трети сериала по подписке
\end{itemize}

\(MNAP@20 = \frac{1}{|U|}\sum_{u\inU}{\frac{1}{min(n_{u},20)}\sum_{i=1}^{20}{r_{u}(i)p_{u}@i\), где
\begin{itemize}
\item \(p_{u}@k=\frac{1}{k}\sum_{i=1}^{k}{r_{u}(i)\)
\item \(r_u{i}\) - потребил ли пользователь u контент, предсказанныйему на месте i (1 либо 0)
\item \(n_{u}\) - количество элементов, которые пользователь потребил за тестовый период
\item \(U\) - множество тестовых пользователей
\end{itemize}

Большинство пользователь досматривают фильмы до конца, поэтому по транзакциям доля положительного класса составляет 65\%. Оценка качества работы алгоритма производилась по подмножеству из 50 тысяч пользователей из представленной выборки.

На момент сбора выборки продуктовая модель рекомендация уже собрана и показывает на тестовой выборке 0.062 NMAP
\section{Эксперименты}
\label{sec:org4033ad3}
\subsection{Описание первого решения}
\label{sec:org98f9083}
В качества начального решения воспользовались моделью K близжайших соседей с расстоянием tf-idf. Модель обучили только на данных про
рейтинг. Получили предсказания для 27\% пользователей из тестовой выборки. Резульатат - 0.0046 NMAP 
\subsection{Некоторые улучшения}
\label{sec:org49c85a9}
Попробовали обучить модели SAR, AlternatingLeastSquare \cite{hu-2008-collab-filter} на полных данных - рейтинг, закладки, транзакции. Лучшая модель модель ALS 200 показала 0.0168 NMAP на тестовой выборке

Также улучшил результаты удаление дублирующихся предсказаний для каждого пользователя и повторное предсказанние удаленных другой
моделью. Уверенность в том, какой из дубликатов удалить делаю на основе предсказаний лучшей модели - получился стекинг SAR и
ALS. Результат - 0.0279 NMAP.

\subsection{Нейросетевой подход}
\label{sec:org799d3a2}
Задачу рекомендации релевантных фильмов и сериалов по историческим данным можно представить как задачу генерации последовательности
чисел. Числа - это идентификаторы фильмов и сериалов. Последовательность идентификаторов отсортирована по дате просмотра.

С такой задачей справляются рекурентные нейросети RNN, LSTM, GRU, CuDNNGRU \cite{fi2020predict}  
\begin{itemize}
\item Было произведена аугментация данных: отсеяны пользователи, у которых в истории меньше 10 просмотренных фильмов, а для оставшихся вся последовательность просмотренных фильмов была разделена на последовательности по 10 фильмов.
\item Задача нейросети - по последовательности фильмов предсказать, какой из 10200 фильмов пользователь посмотрит следующим. Таким образом, получается задача классификации на 10200 классов. В нейросети это сформулировано в виде последнего линейного слоя с размерностью выхода в 10200 и softmax на конце.
\item Функция потерь - кроссэнтропия
\item К каждому фильму применяется обучаемый на этой же выборке эмбединг слой размерностью 128
\item Размерность входных данных: (10, 128)
\item Улучшило результат уменьшение коэффициента скорости обучения в 100 раз при попадании алгоритма на плато в пространстве весов \cite{smith2015cyclical}
\item Получилось предсказать рекомендации для 31 тыс. пользователей из 50 тыс. в тестовой выборке
\item Результат 0.032 NMAP на тестовой выборке
\end{itemize}

\subsubsection{Архитектура нейросети}
\label{sec:org0f1c4ee}
\begin{itemize}
\item Эмбединг слой
\item Пространственный одномерный дропаут
\item Двухслоная CuDNNGRU в 128 нейронов
\item Конкатенация из map pool и avg pool и выхода CuDNNGRU
\item Батчнормализация
\item Выходной линейный слой с softmax
\end{itemize}
\subsection{Ансамблевое решение}
\label{sec:org5cbc917}
\section{Обзор литературы}
\label{sec:orgafcbe47}
  Рекомендательные системы - ключевые технологии для задачи предложения
пользователю релевантный ему контент, услугу или товар на основе интересов
пользователя и его покупательной истории \cite{Isinkaye_2015} . Более чем десяти - летие - возросло использование рекомендательных систем и
глобальные IT корпорации, такие как Google, Facebook, Netflix, Amazon начали
активно использовать рекомендательные системы для увеличения продаж \cite{Jallouli_2017,Jannach_2016,Pathak_2010}. 
Рекомендательные системы разделяют на коллаборативную фильтрацию и контентные
одели.\cite{Isinkaye_2015,10.7838/JSEBS.2012.17.3.085,Son_2015} 


В статье\cite{monti-2018-ensem-approac} описывается подход команды D2KLab к RecSysChallenge 2018, который фокусируется на задаче формирования
плейлиста. Они предложили ансамблевую стратегию различных рекуррентных нейронных сетей, использующих предварительно обученные вложения,
представляющие треки, исполнителей, aльбомов и заголовки в качестве входных данных. Они также использовали тексты песен, из которых они
извлекли семантические и стилистические особенности, которые подали в сеть для творческого трека. RNN изучает вероятностную модель из этих
последовательностей элементов в списке воспроизведения, которая затем используется для прогнозирования наиболее вероятных треков, которые
будут добавлены в список воспроизведения. Что касается плейлистов без треков, то они реализовали резервную стратегию под названием
title2rec, которая генерирует рекомендации, используя только названия песен. Они оптимизировали RNN, Title2Rec и ансамблевый подход для
набора валидации, настраивая гиперпараметры, такие как optimizeralgorithm, скорость обучения и стратегию генерации. Этот подход эффективен
при прогнозировании треков для плейлиста и гибко включает в себя различные типы входных данных, но он также является вычислительно сложным
на этапе обучения.

В \cite{fi2020predict} описывается решение для Rec-Sys Challenge 2019, которое фокусируется на задаче прогнозирования последнего клика в
сеансовом взаимодействии. Команда предложила ансамблевое решение, включающее матричную факторизацию для моделирования взаимодействия пользователя-элемента и сессионно-ориентированную модель обучения, реализованную с помощью рекуррентной нейронной сети. Этот метод, по-видимому, эффективен при прогнозировании последнего клика-аута, набравшего 0,60277 среднего значения в локальном тестовом наборе.



\section{Список литературы}
\label{sec:orgea257f9}
\bibliographystyle{unsrt}
\bibliography{manuscript,manuscript}
\section{Ссылки}
\label{sec:orgbde89b4}
\begin{itemize}
\item \href{http://www.machinelearning.ru/wiki/images/archive/f/fc/20130211221536\%2521Voron-ML-Intro-slides.pdf}{Основные понятия и обозначения в машинном обучении. Воронцов К.В.}
\item \href{https://github.com/esokolov/ml-course-hse/blob/master/2019-fall/lecture-notes/lecture12-factorizations.pdf}{Матричные разложения и рекомендательные системы}
\item \href{https://habr.com/ru/company/tinkoff/blog/454818/}{Решение в Rekko Challenge. 2е место. Блендинг}
\end{itemize}
\end{document}
